% Options for packages loaded elsewhere
\PassOptionsToPackage{unicode}{hyperref}
\PassOptionsToPackage{hyphens}{url}
\documentclass[
]{article}
\usepackage{xcolor}
\usepackage{amsmath,amssymb}
\setcounter{secnumdepth}{-\maxdimen} % remove section numbering
\usepackage{iftex}
\ifPDFTeX
  \usepackage[T1]{fontenc}
  \usepackage[utf8]{inputenc}
  \usepackage{textcomp} % provide euro and other symbols
\else % if luatex or xetex
  \usepackage{unicode-math} % this also loads fontspec
  \defaultfontfeatures{Scale=MatchLowercase}
  \defaultfontfeatures[\rmfamily]{Ligatures=TeX,Scale=1}
\fi
\usepackage{lmodern}
\ifPDFTeX\else
  % xetex/luatex font selection
\fi
% Use upquote if available, for straight quotes in verbatim environments
\IfFileExists{upquote.sty}{\usepackage{upquote}}{}
\IfFileExists{microtype.sty}{% use microtype if available
  \usepackage[]{microtype}
  \UseMicrotypeSet[protrusion]{basicmath} % disable protrusion for tt fonts
}{}
\makeatletter
\@ifundefined{KOMAClassName}{% if non-KOMA class
  \IfFileExists{parskip.sty}{%
    \usepackage{parskip}
  }{% else
    \setlength{\parindent}{0pt}
    \setlength{\parskip}{6pt plus 2pt minus 1pt}}
}{% if KOMA class
  \KOMAoptions{parskip=half}}
\makeatother
\setlength{\emergencystretch}{3em} % prevent overfull lines
\providecommand{\tightlist}{%
  \setlength{\itemsep}{0pt}\setlength{\parskip}{0pt}}
\usepackage{bookmark}
\IfFileExists{xurl.sty}{\usepackage{xurl}}{} % add URL line breaks if available
\urlstyle{same}
\hypersetup{
  pdftitle={More on Finite Automata and Regular Languages},
  pdfauthor={Alyssa Lytle},
  hidelinks,
  pdfcreator={LaTeX via pandoc}}

\title{More on Finite Automata and Regular Languages}
\author{Alyssa Lytle}
\date{January 20, 2026}

\begin{document}
\maketitle

\section{More on Finite Automata and Regular
Languages}\label{more-on-finite-automata-and-regular-languages}

\subsection[Practice ]{\texorpdfstring{Practice
\footnote{\begin{itemize}
  \tightlist
  \item
    Hopcroft, John E., Rajeev Motwani, and Jeffrey D. Ullman.
    ``Introduction to automata theory, languages, and computation.'' Acm
    Sigact News 32.1 (2001): 60-65.
  \end{itemize}}}{Practice }}\label{practice-hopcroft}

Let

\[ L = \{w | w \textrm{ is of even length and begins with } 01 \}\]

Prove \(L\) is a regular language.

\subsection{Practice}\label{practice}

Let us design a DFA to accept the language:

\[ L = \{w | w \textrm{ is of even length and begins with } 01 \}\]

What we need to track:

\begin{itemize}
\tightlist
\item
  Whether it starts with 01
\item
  Whether the input length is even
\end{itemize}

\subsection{More notation}\label{more-notation}

Let's talk more about our transition function \(\delta\)\ldots{}

We've so far discussed \(\delta\) in terms of

\begin{itemize}
\tightlist
\item
  \(\delta: Q \times \Sigma \to Q\)
\end{itemize}

In other words \(\delta\) is handling state transitions given a single
input from \(\Sigma\).

What if we wanted to discuss \emph{strings} of input?

\subsection{Extending the Transition Function to
Strings}\label{extending-the-transition-function-to-strings}

\[\Hat{\delta}: Q \times \Sigma^* \to Q\]

\subsubsection{Formal Definition}\label{formal-definition}

Let \(w\) be a string of the form \(xa\);

That is \(w= xa\), where \(x\) is a string and \(a\) is a symbol.

Then

\[\Hat{\delta}(q,w) = \delta(\Hat{\delta}(q,x),a)\]

\subsection{Practice}\label{practice-1}

For the DFA we previously defined, show
\(\Hat{\delta}(q_0, 011101) = q_2\)

\subsection[Proving Properties About Automata ]{\texorpdfstring{Proving
Properties About Automata \footnote{\begin{itemize}
  \tightlist
  \item
    Sipser, Michael. ``Introduction to the Theory of Computation.'' ACM
    Sigact News 27.1 (1996): 27-29
  \end{itemize}}
\footnote{\begin{itemize}
  \tightlist
  \item
    Kozen, Dexter C. Automata and computability. Springer Science \&
    Business Media, 2007.
  \end{itemize}}}{Proving Properties About Automata  }}\label{proving-properties-about-automata-sipser-kozen}

\subsubsection{Want to Prove (WTP):}\label{want-to-prove-wtp}

Assume \(A\) and \(B\) are regular languages. Then \(A \cap B\) is also
regular.

\subsection{Proving Properties About
Automata}\label{proving-properties-about-automata}

\subsubsection{Want to Prove (WTP):}\label{want-to-prove-wtp-1}

Assume \(A\) and \(B\) are regular languages. Then \(A \cap B\) is also
regular.

If \(A\) and \(B\) are regular, then there exist automata

\(M_1 = (Q_1, \Sigma, \delta_1, s_1, F_1)\)

\(M_2 = (Q_2, \Sigma, \delta_2, s_2, F_2)\)

With \(L(M_1) = A\) and \(L(M_2) = B\).

To prove \(A \cap B\) is regular, we need to build an automaton for it!

\emph{It's a proof by construction!}

\subsection{Proof by Construction}\label{proof-by-construction}

Has two main elements:

\begin{enumerate}
\def\labelenumi{\arabic{enumi}.}
\tightlist
\item
  The construction
\item
  Proof that the construction satisfies the claim
\end{enumerate}

\subsection{Proof by Construction}\label{proof-by-construction-1}

Has two main elements:

\begin{enumerate}
\def\labelenumi{\arabic{enumi}.}
\tightlist
\item
  Construction of automaton \(M_3\)
\item
  Proof that \(L(M_3) = L(M_1) \cap L(M_2)\)
\end{enumerate}

\subsection{\texorpdfstring{Construction of
\(M_3\)}{Construction of M\_3}}\label{construction-of-m_3}

\begin{itemize}
\tightlist
\item
  \(L(M_1) = A\) and \(L(M_2) = B\).
\item
  \(M_1 = (Q_1, \Sigma, \delta_1, s_1, F_1)\)
\item
  \(M_2 = (Q_2, \Sigma, \delta_2, s_2, F_2)\)
\end{itemize}

Let \(M_3 = (Q_3, \Sigma, \delta_3, s_3, F_3)\)

with

\subsection{\texorpdfstring{Proof that
\(L(M_3) = L(M_1) \cap L(M_2)\)}{Proof that L(M\_3) = L(M\_1) \textbackslash cap L(M\_2)}}\label{proof-that-lm_3-lm_1-cap-lm_2}

We're going to work with this assumption:

\subsubsection{Lemma A}\label{lemma-a}

For all \(x \in \Sigma^*\),
\(\Hat{\delta_3}((p,q),x) = (\Hat{\delta_1}(p,x),\Hat{\delta_2}(q,x))\).

\subsection{\texorpdfstring{Proof that
\(L(M_3) = L(M_1) \cap L(M_2)\)}{Proof that L(M\_3) = L(M\_1) \textbackslash cap L(M\_2)}}\label{proof-that-lm_3-lm_1-cap-lm_2-1}

\subsubsection{Our Toolbox (What We
Know)}\label{our-toolbox-what-we-know}

\begin{itemize}
\tightlist
\item
  \(L(M_1) = A\) and \(L(M_2) = B\).
\item
  \(M_1 = (Q_1, \Sigma, \delta_1, s_1, F_1)\)
\item
  \(M_2 = (Q_2, \Sigma, \delta_2, s_2, F_2)\)
\item
  Our Construction of \(M_3= (Q_3, \Sigma, \delta_3, s_3, F_3)\)

  \begin{itemize}
  \tightlist
  \item
    \(Q_3 = Q_1 \times Q_2 = \{(p,q) \mid p \in Q_1, q \in Q_2 \}\)
  \item
    \(\delta_3((p,q),d) = (\delta_1(p,d), \delta_2(q,d))\)
  \item
    \(s_3 = (s_1,s_2)\)
  \item
    \(F_3 = F_1 \times F_2\)
  \end{itemize}
\item
  Lemma A:
  \(\Hat{\delta_3}((p,q),x) = (\Hat{\delta_1}(p,x),\Hat{\delta_2}(q,x))\).
\end{itemize}

\subsection{Resources}\label{resources}

\end{document}
