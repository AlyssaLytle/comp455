<?xml version="1.0" encoding="utf-8"?>
<!DOCTYPE html PUBLIC "-//W3C//DTD XHTML 1.0 Strict//EN"
 "http://www.w3.org/TR/xhtml1/DTD/xhtml1-strict.dtd">
<html xmlns="http://www.w3.org/1999/xhtml">
<head>
  <meta http-equiv="Content-Type" content="text/html; charset=utf-8" />
  <meta http-equiv="Content-Style-Type" content="text/css" />
  <meta name="generator" content="pandoc" />
  <meta name="author" content="Alyssa Lytle" />
  <meta name="date" content="2025-08-21" />
  <title>Finite Automata and Regular Languages</title>
  <style type="text/css">
    /* Default styles provided by pandoc.
    ** See https://pandoc.org/MANUAL.html#variables-for-html for config info.
    */
    code{white-space: pre-wrap;}
    span.smallcaps{font-variant: small-caps;}
    div.columns{display: flex; gap: min(4vw, 1.5em);}
    div.column{flex: auto; overflow-x: auto;}
    div.hanging-indent{margin-left: 1.5em; text-indent: -1.5em;}
    /* The extra [class] is a hack that increases specificity enough to
       override a similar rule in reveal.js */
    ul.task-list[class]{list-style: none;}
    ul.task-list li input[type="checkbox"] {
      font-size: inherit;
      width: 0.8em;
      margin: 0 0.8em 0.2em -1.6em;
      vertical-align: middle;
    }
    .display.math{display: block; text-align: center; margin: 0.5rem auto;}
  </style>
  <link rel="stylesheet" type="text/css" media="screen, projection, print"
    href="https://www.w3.org/Talks/Tools/Slidy2/styles/slidy.css" />
  <script src="https://www.w3.org/Talks/Tools/Slidy2/scripts/slidy.js"
    charset="utf-8" type="text/javascript"></script>
</head>
<body>
<div class="slide titlepage">
  <h1 class="title">Finite Automata and Regular Languages</h1>
  <p class="author">
Alyssa Lytle
  </p>
  <p class="date">August 21, 2025</p>
</div>
<div class="slide section level2">

<!-- pandoc -t slidy -s notes/01-fa.md -o slides/03-finite-automata.html --webtex -->
</div>
<div id="finite-automata-and-regular-languages"
class="title-slide slide section level1">
<h1>Finite Automata and Regular Languages</h1>

</div>
<div id="finite-automaton-an-example" class="slide section level2">
<h1>Finite Automaton: An Example</h1>
<p>Our example: an automatic door</p>
<p><img src="../static/slide_figs/auto-door.png" /><a href="#fn1"
class="footnote-ref" id="fnref1"><sup>1</sup></a></p>
<p>Front pad: detect person to walk through</p>
<p>Rear pad: Confirm person has passed through, don’t hit other person
standing there</p>
</div>
<div id="rules-of-operation" class="slide section level2">
<h1>Rules of Operation</h1>
<table>
<thead>
<tr>
<th></th>
<th align="center">Front</th>
<th align="center">Rear</th>
<th align="center">Both</th>
<th align="center">Neither</th>
</tr>
</thead>
<tbody>
<tr>
<td><strong>Closed</strong></td>
<td align="center"></td>
<td align="center"></td>
<td align="center"></td>
<td align="center"></td>
</tr>
<tr>
<td><strong>Open</strong></td>
<td align="center"></td>
<td align="center"></td>
<td align="center"></td>
<td align="center"></td>
</tr>
</tbody>
</table>
</div>
<div id="rules-of-operation-1" class="slide section level2">
<h1>Rules of Operation</h1>
<table>
<thead>
<tr>
<th></th>
<th align="center">Front</th>
<th align="center">Rear</th>
<th align="center">Both</th>
<th align="center">Neither</th>
</tr>
</thead>
<tbody>
<tr>
<td><strong>Closed</strong></td>
<td align="center">Open</td>
<td align="center">Closed</td>
<td align="center">Closed</td>
<td align="center">Closed</td>
</tr>
<tr>
<td><strong>Open</strong></td>
<td align="center">Open</td>
<td align="center">Open</td>
<td align="center">Open</td>
<td align="center">Closed</td>
</tr>
</tbody>
</table>
<p>This is called a <strong>state transition table</strong>.</p>
</div>
<div id="state-diagram" class="slide section level2">
<h1>State Diagram</h1>
<p><img src="../static/slide_figs/autodoor-informal.png" /></p>
<p>This is called a state diagram.</p>
</div>
<div id="formal-requirements" class="slide section level2">
<h1>Formal Requirements</h1>
<p>A finite automaton should have</p>
<ul>
<li>A finite set of states</li>
<li>A finite alphabet</li>
<li>A transition function</li>
<li>A start state</li>
<li>A set of accept states</li>
</ul>
</div>
<div id="formal-definition" class="slide section level2">
<h1>Formal Definition</h1>
<p>A finite automaton can be expressed as 5-tuple <span
class="math inline">(<em>Q</em>, <em>Σ</em>, <em>δ</em>, <em>q</em><sub>0</sub>, <em>F</em>)</span>
where:</p>
<ul>
<li><span class="math inline"><em>Q</em></span>: A finite set of
states</li>
<li><span class="math inline"><em>Σ</em></span>: A finite alphabet</li>
<li><span
class="math inline"><em>δ</em> : <em>Q</em> × <em>Σ</em> → <em>Q</em></span>:
A transition function</li>
<li><span
class="math inline"><em>q</em><sub>0</sub> ∈ <em>Q</em></span>: A start
state</li>
<li><span class="math inline"><em>F</em> ⊆ <em>Q</em></span>: A set of
accept states</li>
</ul>
</div>
<div id="if-were-being-technical-about-it" class="slide section level2">
<h1>If We’re Being Technical About It…</h1>
<p>(Which we are!)</p>
<p><img src="../static/slide_figs/formal-door.png" /></p>
<ul>
<li><span
class="math inline"><em>Q</em> : {<em>O</em><em>p</em><em>e</em><em>n</em>, <em>C</em><em>l</em><em>o</em><em>s</em><em>e</em><em>d</em>}</span></li>
<li><span
class="math inline"><em>Σ</em> : {<em>F</em><em>r</em><em>o</em><em>n</em><em>t</em>, <em>R</em><em>e</em><em>a</em><em>r</em>, <em>B</em><em>o</em><em>t</em><em>h</em>, <em>N</em><em>e</em><em>i</em><em>t</em><em>h</em><em>e</em><em>r</em>}</span></li>
<li><span
class="math inline"><em>δ</em> : <em>Q</em> × <em>Σ</em> → <em>Q</em> : Our
transition table</span></li>
<li><span
class="math inline"><em>q</em><sub>0</sub> ∈ <em>Q</em> : <em>C</em><em>l</em><em>o</em><em>s</em><em>e</em><em>d</em></span></li>
<li><span
class="math inline"><em>F</em> ⊆ <em>Q</em> : {<em>O</em><em>p</em><em>e</em><em>n</em>, <em>C</em><em>l</em><em>o</em><em>s</em><em>e</em><em>d</em>}</span></li>
</ul>
</div>
<div id="the-language-of-the-machine" class="slide section level2">
<h1>The Language of the Machine</h1>
<p><img src="../static/slide_figs/formal-door.png" /></p>
<p>For our machine <span class="math inline"><em>M</em></span>, The
<em>language</em> of our machine <span
class="math inline"><em>L</em>(<em>M</em>)</span> is the set of all
string inputs that <span class="math inline"><em>M</em></span>
accepts…</p>
<p>E.g. <span
class="math inline">{<em>ϵ</em>, <em>F</em><em>r</em><em>o</em><em>n</em><em>t</em>, <em>F</em><em>r</em><em>o</em><em>n</em><em>t</em><em>R</em><em>e</em><em>a</em><em>r</em>, <em>F</em><em>r</em><em>o</em><em>n</em><em>t</em><em>R</em><em>e</em><em>a</em><em>r</em><em>F</em><em>r</em><em>o</em><em>n</em><em>t</em>, <em>R</em><em>e</em><em>a</em><em>r</em><em>B</em><em>o</em><em>t</em><em>h</em>, ...}</span></p>
<p><span class="math inline"><em>M</em></span> accepts or
<em>recognizes</em> a string if it terminates in an accept state.</p>
<p>(This isn’t the best example because this door accepts all input
strings, so let’s try another one!)</p>
</div>
<div id="another-example" class="slide section level2">
<h1>Another Example</h1>
<p>Let <span class="math inline"><em>M</em></span> be:</p>
<p><img src="../static/slide_figs/fa.png" /><a href="#fn2"
class="footnote-ref" id="fnref2"><sup>2</sup></a></p>
<p>What are <span
class="math inline">(<em>Q</em>, <em>Σ</em>, <em>δ</em>, <em>q</em><sub>0</sub>, <em>F</em>)</span>?</p>
</div>
<div id="another-example-1" class="slide section level2">
<h1>Another Example</h1>
<p>Let <span class="math inline"><em>M</em></span> be:</p>
<p><img src="../static/slide_figs/fa.png" /><a href="#fn3"
class="footnote-ref" id="fnref3"><sup>3</sup></a></p>
<p>What is <span class="math inline"><em>L</em>(<em>M</em>)</span>?</p>
</div>
<div id="formal-definition-of-computationsipser"
class="slide section level2">
<h1>Formal Definition of Computation<a href="#fn4" class="footnote-ref"
id="fnref4"><sup>4</sup></a></h1>
<p>Let <span
class="math inline"><em>M</em> = (<em>Q</em>, <em>Σ</em>, <em>δ</em>, <em>q</em><sub>0</sub>, <em>F</em>)</span></p>
<p>Let <span
class="math inline"><em>w</em> = <em>w</em><sub>1</sub><em>w</em><sub>2</sub>…<em>w</em><sub><em>n</em></sub></span>
be a string where each <span
class="math inline"><em>w</em><sub><em>i</em></sub></span> is a member
of the alphabet <span class="math inline"><em>Σ</em></span></p>
<p><span class="math inline"><em>M</em></span> accepts <span
class="math inline"><em>w</em></span> if there exists a sequence of
<em>states</em> <span
class="math inline"><em>r</em><sub>1</sub>, <em>r</em><sub>2</sub>, …, <em>r</em><sub><em>n</em></sub></span>
such that:</p>
<ol style="list-style-type: decimal">
<li><span
class="math inline"><em>r</em><sub>0</sub> = <em>q</em><sub>0</sub></span></li>
<li><span
class="math inline"><em>δ</em>(<em>r</em><sub><em>i</em></sub>, <em>w</em><sub><em>i</em> + 1</sub>) = <em>r</em><sub><em>i</em> + 1</sub></span>
for <span class="math inline"><em>i</em> = 0, …, <em>n</em> − 1</span>,
and</li>
<li><span
class="math inline"><em>r</em><sub><em>n</em></sub> ∈ <em>F</em></span></li>
</ol>
</div>
<div id="a-regular-language" class="slide section level2">
<h1>A Regular Language</h1>
<p>A language is a <em>regular language</em> if there exists a finite
automaton that recognizes it.</p>
</div>
<div id="language-operations" class="slide section level2">
<h1>Language Operations</h1>
<ul>
<li><p>Union: <span
class="math inline"><em>A</em> ∪ <em>B</em> = {<em>x</em>|<em>x</em> ∈ <em>A</em>
or <em>x</em> ∈ <em>B</em>}</span></p></li>
<li><p>Concatenation: <span
class="math inline"><em>A</em> ∘ <em>B</em> = {<em>x</em><em>y</em>|<em>x</em> ∈ <em>A</em>
and <em>y</em> ∈ <em>B</em>}</span></p></li>
<li><p>Star: <span
class="math inline"><em>A</em><sup>*</sup> = {<em>x</em><sub>1</sub><em>x</em><sub>2</sub>…<em>x</em><sub><em>k</em></sub>|<em>k</em> ≥ 0
and each <em>x</em><sub><em>i</em></sub> ∈ <em>A</em>}</span></p></li>
</ul>
</div>
<div id="language-operations-example" class="slide section level2">
<h1>Language Operations Example</h1>
<p>Let <span class="math inline"><em>Σ</em></span> be the standard
English alphabet</p>
<p>If <span class="math inline"><em>A</em> = {good, bad}</span> <span
class="math inline"><em>B</em> = {cat, dog}</span></p>
<ul>
<li><p><span class="math inline"><em>A</em> ∪ <em>B</em></span></p></li>
<li><p><span class="math inline"><em>A</em> ∘ <em>B</em></span></p></li>
<li><p><span class="math inline"><em>A</em><sup>*</sup></span></p></li>
</ul>
</div>
<div id="resources" class="slide section level2">
<h1>Resources</h1>
</div>

<div class="footnotes slide">
<hr />
<ol>
<li id="fn1"><p>Sipser, Michael. “Introduction to the Theory of
Computation.” ACM Sigact News 27.1 (1996): 27-29.<a href="#fnref1"
class="footnote-back">↩︎</a></p></li>
<li id="fn2"><p>Sipser, Michael. “Introduction to the Theory of
Computation.” ACM Sigact News 27.1 (1996): 27-29.<a href="#fnref2"
class="footnote-back">↩︎</a></p></li>
<li id="fn3"><p>Sipser, Michael. “Introduction to the Theory of
Computation.” ACM Sigact News 27.1 (1996): 27-29.<a href="#fnref3"
class="footnote-back">↩︎</a></p></li>
<li id="fn4"><p>Sipser, Michael. “Introduction to the Theory of
Computation.” ACM Sigact News 27.1 (1996): 27-29.<a href="#fnref4"
class="footnote-back">↩︎</a></p></li>
</ol>
</div>
</body>
</html>
