% Options for packages loaded elsewhere
\PassOptionsToPackage{unicode}{hyperref}
\PassOptionsToPackage{hyphens}{url}
\documentclass[
]{article}
\usepackage{xcolor}
\usepackage{amsmath,amssymb}
\setcounter{secnumdepth}{-\maxdimen} % remove section numbering
\usepackage{iftex}
\ifPDFTeX
  \usepackage[T1]{fontenc}
  \usepackage[utf8]{inputenc}
  \usepackage{textcomp} % provide euro and other symbols
\else % if luatex or xetex
  \usepackage{unicode-math} % this also loads fontspec
  \defaultfontfeatures{Scale=MatchLowercase}
  \defaultfontfeatures[\rmfamily]{Ligatures=TeX,Scale=1}
\fi
\usepackage{lmodern}
\ifPDFTeX\else
  % xetex/luatex font selection
\fi
% Use upquote if available, for straight quotes in verbatim environments
\IfFileExists{upquote.sty}{\usepackage{upquote}}{}
\IfFileExists{microtype.sty}{% use microtype if available
  \usepackage[]{microtype}
  \UseMicrotypeSet[protrusion]{basicmath} % disable protrusion for tt fonts
}{}
\makeatletter
\@ifundefined{KOMAClassName}{% if non-KOMA class
  \IfFileExists{parskip.sty}{%
    \usepackage{parskip}
  }{% else
    \setlength{\parindent}{0pt}
    \setlength{\parskip}{6pt plus 2pt minus 1pt}}
}{% if KOMA class
  \KOMAoptions{parskip=half}}
\makeatother
\usepackage{longtable,booktabs,array}
\usepackage{calc} % for calculating minipage widths
% Correct order of tables after \paragraph or \subparagraph
\usepackage{etoolbox}
\makeatletter
\patchcmd\longtable{\par}{\if@noskipsec\mbox{}\fi\par}{}{}
\makeatother
% Allow footnotes in longtable head/foot
\IfFileExists{footnotehyper.sty}{\usepackage{footnotehyper}}{\usepackage{footnote}}
\makesavenoteenv{longtable}
\usepackage{graphicx}
\makeatletter
\newsavebox\pandoc@box
\newcommand*\pandocbounded[1]{% scales image to fit in text height/width
  \sbox\pandoc@box{#1}%
  \Gscale@div\@tempa{\textheight}{\dimexpr\ht\pandoc@box+\dp\pandoc@box\relax}%
  \Gscale@div\@tempb{\linewidth}{\wd\pandoc@box}%
  \ifdim\@tempb\p@<\@tempa\p@\let\@tempa\@tempb\fi% select the smaller of both
  \ifdim\@tempa\p@<\p@\scalebox{\@tempa}{\usebox\pandoc@box}%
  \else\usebox{\pandoc@box}%
  \fi%
}
% Set default figure placement to htbp
\def\fps@figure{htbp}
\makeatother
\setlength{\emergencystretch}{3em} % prevent overfull lines
\providecommand{\tightlist}{%
  \setlength{\itemsep}{0pt}\setlength{\parskip}{0pt}}
\usepackage{bookmark}
\IfFileExists{xurl.sty}{\usepackage{xurl}}{} % add URL line breaks if available
\urlstyle{same}
\hypersetup{
  pdftitle={Finite Automata and Regular Languages},
  pdfauthor={Alyssa Lytle},
  hidelinks,
  pdfcreator={LaTeX via pandoc}}

\title{Finite Automata and Regular Languages}
\author{Alyssa Lytle}
\date{August 21, 2025}

\begin{document}
\maketitle

\section{Finite Automata and Regular
Languages}\label{finite-automata-and-regular-languages}

\subsection{Finite Automaton: An
Example}\label{finite-automaton-an-example}

Our example: an automatic door

\pandocbounded{\includegraphics[keepaspectratio]{../static/slide_figs/auto-door.png}}\footnote{Sipser,
  Michael. ``Introduction to the Theory of Computation.'' ACM Sigact
  News 27.1 (1996): 27-29.}

Front pad: detect person to walk through

Rear pad: Confirm person has passed through, don't hit other person
standing there

\subsection{Rules of Operation}\label{rules-of-operation}

{\def\LTcaptype{} % do not increment counter
\begin{longtable}[]{@{}lcccc@{}}
\toprule\noalign{}
& Front & Rear & Both & Neither \\
\midrule\noalign{}
\endhead
\bottomrule\noalign{}
\endlastfoot
\textbf{Closed} & & & & \\
\textbf{Open} & & & & \\
\end{longtable}
}

\subsection{Rules of Operation}\label{rules-of-operation-1}

{\def\LTcaptype{} % do not increment counter
\begin{longtable}[]{@{}lcccc@{}}
\toprule\noalign{}
& Front & Rear & Both & Neither \\
\midrule\noalign{}
\endhead
\bottomrule\noalign{}
\endlastfoot
\textbf{Closed} & Open & Closed & Closed & Closed \\
\textbf{Open} & Open & Open & Open & Closed \\
\end{longtable}
}

This is called a \textbf{state transition table}.

\subsection{State Diagram}\label{state-diagram}

\pandocbounded{\includegraphics[keepaspectratio]{../static/slide_figs/autodoor-informal.png}}

This is called a state diagram.

\subsection{Formal Requirements}\label{formal-requirements}

A finite automaton should have

\begin{itemize}
\tightlist
\item
  A finite set of states
\item
  A finite alphabet
\item
  A transition function
\item
  A start state
\item
  A set of accept states
\end{itemize}

\subsection{Formal Definition}\label{formal-definition}

A finite automaton can be expressed as 5-tuple
\((Q, \Sigma, \delta,q_0, F)\) where:

\begin{itemize}
\tightlist
\item
  \(Q\): A finite set of states
\item
  \(\Sigma\): A finite alphabet
\item
  \(\delta: Q \times \Sigma \to Q\): A transition function
\item
  \(q_0 \in Q\): A start state
\item
  \(F \subseteq Q\): A set of accept states
\end{itemize}

\subsection{If We're Being Technical About
It\ldots{}}\label{if-were-being-technical-about-it}

(Which we are!)

\pandocbounded{\includegraphics[keepaspectratio]{../static/slide_figs/formal-door.png}}

\begin{itemize}
\tightlist
\item
  \(Q: \{Open, Closed\}\)
\item
  \(\Sigma: \{Front, Rear, Both, Neither\}\)
\item
  \(\delta: Q \times \Sigma \to Q: \textrm{Our transition table}\)
\item
  \(q_0 \in Q: Closed\)
\item
  \(F \subseteq Q: \{Open, Closed\}\)
\end{itemize}

\subsection{The Language of the
Machine}\label{the-language-of-the-machine}

\pandocbounded{\includegraphics[keepaspectratio]{../static/slide_figs/formal-door.png}}

For our machine \(M\), The \emph{language} of our machine \(L(M)\) is
the set of all string inputs that \(M\) accepts\ldots{}

E.g. \(\{\epsilon, Front, FrontRear, FrontRearFront, RearBoth, ...\}\)

\(M\) accepts or \emph{recognizes} a string if it terminates in an
accept state.

(This isn't the best example because this door accepts all input
strings, so let's try another one!)

\subsection{Another Example}\label{another-example}

Let \(M\) be:

\pandocbounded{\includegraphics[keepaspectratio]{../static/slide_figs/fa.png}}\footnote{Sipser,
  Michael. ``Introduction to the Theory of Computation.'' ACM Sigact
  News 27.1 (1996): 27-29.}

What are \((Q, \Sigma, \delta,q_0, F)\)?

\subsection{Another Example}\label{another-example-1}

Let \(M\) be:

\pandocbounded{\includegraphics[keepaspectratio]{../static/slide_figs/fa.png}}\footnote{Sipser,
  Michael. ``Introduction to the Theory of Computation.'' ACM Sigact
  News 27.1 (1996): 27-29.}

What is \(L(M)\)?

\subsection[Formal Definition of Computation]{\texorpdfstring{Formal
Definition of
Computation\footnote{Sipser, Michael. ``Introduction to the Theory of
  Computation.'' ACM Sigact News 27.1 (1996): 27-29.}}{Formal Definition of Computation}}\label{formal-definition-of-computationsipser}

Let \(M = (Q, \Sigma, \delta,q_0, F)\)

Let \(w = w_1w_2\ldots w_n\) be a string where each \(w_i\) is a member
of the alphabet \(\Sigma\)

\(M\) accepts \(w\) if there exists a sequence of \emph{states}
\(r_1, r_2, \ldots, r_n\) such that:

\begin{enumerate}
\def\labelenumi{\arabic{enumi}.}
\tightlist
\item
  \(r_0 = q_0\)
\item
  \(\delta(r_i,w_{i+1})= r_{i+1}\) for \(i = 0,\ldots,n-1\), and
\item
  \(r_n \in F\)
\end{enumerate}

\subsection{A Regular Language}\label{a-regular-language}

A language is a \emph{regular language} if there exists a finite
automaton that recognizes it.

\subsection{Language Operations}\label{language-operations}

\begin{itemize}
\item
  Union: \(A \cup B = \{x| x\in A \textrm{ or } x \in B\}\)
\item
  Concatenation: \(A \circ B = \{xy| x \in A \textrm{ and } y \in B\}\)
\item
  Star:
  \(A^* = \{x_1 x_2 \ldots x_k | k \geq 0 \textrm{ and each } x_i \in A \}\)
\end{itemize}

\subsection{Language Operations
Example}\label{language-operations-example}

Let \(\Sigma\) be the standard English alphabet

If \(A = \{\textrm{good}, \textrm{bad}\}\)
\(B= \{\textrm{cat}, \textrm{dog}\}\)

\begin{itemize}
\item
  \(A \cup B\)
\item
  \(A \circ B\)
\item
  \(A^*\)
\end{itemize}

\subsection{Gumball Machine Problem}\label{gumball-machine-problem}

\begin{verbatim}
Design a DFA that represents a gumball machine with the following properties:

* It takes nickels and dimes as inputs
* If it receives 15 cents total, it dispenses a gumball
* If it receives more than 15 cents, it dispenses a gumball and change


Think of an "accept" state as one where a gumball is dispensed.

(It's ok if your solution doesn't look quite like your neighbor's! There are multiple correct answers! We're going to compare!)
\end{verbatim}

\subsection{Gumball Machine Solutions}\label{gumball-machine-solutions}

\subsection{Resources}\label{resources}

\end{document}
