\subsection{Proofs}\label{proofs}

\begin{itemize}
\tightlist
\item
  Proofs are useful because they let you say something about the tools
  you're using and building!
\item
  In the real world, this is useful because it helps you explain your
  work and really convince people of your contribution.
\item
  Proofs are also helpful because they force you to write down your
  thoughts. In writing down your thoughts, you might find out that your
  thinking is flawed or you're making assumptions where you shouldn't.\\
\item
  Examples of some things that you might prove:

  \begin{itemize}
  \tightlist
  \item
    This cryptography scheme is secure.
  \item
    This new algorithm is the fastest out there.
  \item
    This network will lose less than \(.001 \%\) of the messages sent
    over it.
  \item
    This new robot won't become sentient and kill you.
  \end{itemize}
\end{itemize}

\subsection{What a proof looks like}\label{what-a-proof-looks-like}

\begin{itemize}
\tightlist
\item
  A good proof is like good code, clear and easy to follow!
\item
  Think of narrating your proof like commenting your code--you want the
  next person to read it to be able to follow your train of thought.
\item
  I'm going to introduce a specific format for writing proofs. \emph{You
  do not have to use this format for this course}, but it can be useful
  to make sure you are fully documenting and explaining your proofs. It
  also gives a linear structure to your argument.
\end{itemize}

\subsection{What a proof looks like}\label{what-a-proof-looks-like-1}

The two column format looks like this:

\(\begin{array}{l l l}
 & \textrm{Column 1} & \textrm{Column 2} \\
 \hline
    1. & \textrm{Your conclusion} & \textrm{How you came to that conclusion}  \\
     2. & \ldots\\
     3. & \ldots\\
     \vdots \\
      \square
 \end{array}\)

\subsection{Finding P and Q}\label{finding-p-and-q}

\subsubsection{Example}\label{example}

\begin{itemize}
\tightlist
\item
  ``If \(x\) is even, \(x^2\) is even.''
\item
  \(P:\) \(x\) is even
\item
  \(Q:\) \(x^2\) is even
\end{itemize}

\subsection{Finding P and Q}\label{finding-p-and-q-1}

\subsubsection{Example}\label{example-1}

\begin{itemize}
\tightlist
\item
  ``There are infinitely many primes.''
\item
  This can be restated as ``For all \(a \in \mathbb{Z}\), if \(a\) is
  prime, then there exists a prime \(b \in \mathbb{Z}\) such that
  \(b>a\).'\,'\\
\item
  \(P:\) \(a \in \mathbb{Z}\) is prime
\item
  \(Q:\) There exists \(b \in \mathbb{Z}\) such that \(b>a\)
\end{itemize}

\subsection{Finding P and Q}\label{finding-p-and-q-2}

\subsubsection{Example}\label{example-2}

\begin{itemize}
\tightlist
\item
  ``\(x-1 < \left \lfloor{x}\right \rfloor\)''
\item
  \(P:\) Everything you know about math is true
\item
  \(Q:\) \(x-1 < \left \lfloor{x}\right \rfloor\)
\end{itemize}

\subsection{Rewrites}\label{rewrites}

Here are some common rewrites you might see when constructing your
proofs:

\begin{itemize}
\tightlist
\item
  \(n\) is an even integer converts to \(n = 2t\) for some \(t\)
\item
  \(n\) is an odd integer converts to \(n = 2t + 1\) for some \(t\)
\item
  \(n\) is a rational number converts to \(n = \frac{a}{b}\) where \(a\)
  and \(b\) are integers
\item
  \(n\) divides \(m\) converts to \(m = nt\) for some integer \(t\)
\item
  \(n\) is a square converts to \(n = t^2\) for some integer \(t\).
\item
  \(n = a \bmod b\) converts to \(n = bt + a\) for some integer \(t\).
\end{itemize}

\subsection{Main Types of Proofs}\label{main-types-of-proofs}

The four typical ways we show \(P \implies Q\) are the following:

\begin{itemize}
\item
  Direct Proof
\item
  Proof By Contradiction
\item
  Proof By Contrapositive
\item
  Proof By Counterexample
\item
  Proof By Induction
\end{itemize}

\subsection{Types of Proofs - Direct}\label{types-of-proofs---direct}

The first type of proof we will discuss is called a \textbf{direct
proof}. Basically, we are trying to to prove \(p  \rightarrow q\) by
starting at \(p\) and getting to \(q\).

For example, say we want to prove the following: ``If \(x\) is even,
\(x^2\) is even''.

Then \(p\) is ``\(x\) is even'' and \(q\) is ``\(x^2\) is even''

\subsection{Types of Proofs - Direct}\label{types-of-proofs---direct-1}

The first type of proof we will discuss is called a \textbf{direct
proof}. Basically, we are trying to to prove \(p  \rightarrow q\) by
starting at \(p\) and getting to \(q\).

For example, say we want to prove the following: ``If \(x\) is even,
\(x^2\) is even''.

Then \(p\) is ``\(x\) is even'' and \(q\) is ``\(x^2\) is even''

\(\begin{array}{l l l}
      1. & x \textrm{ is even} & \textrm{Given (In other words, this is our } p \textrm{)}\\
      2. & x = 2k, {} k \in \mathbb{Z} & \textrm{Definition of Even Number}\\
      3. & x^2 = (2k)^2 & \textrm{Squared both sides of Line 2} \\
      4. & x^2 = 4k^2 & \textrm{Simplified Line 3} \\
      5. & x^2 = 2(2k^2) & \textrm{Rewrote Line 4} \\
      6. & x^2 \textrm{ is even} & \textrm{Definition of Even Number } \square
    \end{array}\)

\textbf{Note that at every step you're basically saying,
``Therefore\ldots{}'' this is where your implication \(\rightarrow\) is
coming in.}

\subsection{Proof By Contradiction}\label{proof-by-contradiction}

\begin{itemize}
\item
  \(P \implies Q \equiv \neg(P \land \neg Q)\).
\item
  This leads to why we do proofs by contradiction.
\item
  Instead of proving \(P \implies Q\), we prove \(P \land \neg Q\) does
  not hold.
\item
  In other words, we start with \(P \land \neg Q\) and get to an
  impossible state \(\bot\).
\end{itemize}

\subsubsection{Example}\label{example-3}

\begin{itemize}
\tightlist
\item
  A famous proof is the proof that \(\sqrt{2}\) is irrational.
\end{itemize}

\subsection{Proof By Contradiction -
Example}\label{proof-by-contradiction---example}

\begin{itemize}
\tightlist
\item
  We want to prove: If \(x^2\) is odd, \(x\) is odd.
\end{itemize}

\subsection{Proof by Contrapositive}\label{proof-by-contrapositive}

For a proof of contrapositive, you're going to use the equality we
proved last class:

\(P \implies Q \equiv \neg Q \implies \neg P\).

In other words, instead of proving \(P \implies Q\), we are going to
assume \(\neg Q\) and show that it leads to \(\neg P\).

\subsubsection{Example}\label{example-4}

\begin{itemize}
\tightlist
\item
  We want to prove: If \(x^2\) is odd, \(x\) is odd.
\end{itemize}

\subsection{Proof by Counterexample}\label{proof-by-counterexample}

If you want to disprove something, the easiest way is usually by counter
example.

You don't have to do this in the typical two column format as long as
you make your reasoning clear.

\subsubsection{Example}\label{example-5}

\begin{itemize}
\tightlist
\item
  Say I ask you to prove the following is false: If \(x\) is even,
  \(x^2\) is odd.
\end{itemize}

\subsection{Other Types of Proofs}\label{other-types-of-proofs}

You may see some other types of proofs that follow from the types of
proofs we've already discussed.

\begin{itemize}
\tightlist
\item
  Forall

  \begin{itemize}
  \tightlist
  \item
    Exhaustive

    \begin{itemize}
    \tightlist
    \item
      Proof by Cases
    \end{itemize}
  \end{itemize}
\item
  Existence

  \begin{itemize}
  \tightlist
  \item
    Proof by Construction
  \end{itemize}
\end{itemize}

\subsection{Forall}\label{forall}

You may want to prove statements such as
``\(\forall x \in S, P(x) \implies Q(x)\).''

As previously discussed, to prove a ``for all'' statement, you want to
look at every element in \(S\) and show that \(P(x) \implies Q(x)\)
holds.

You can often do this with a direct proof. All of the things we've
directly proved so far are actually ``for all'' proofs.

\begin{itemize}
\tightlist
\item
  \(\forall x \in \mathbb{Z}\), If \(x\) is even, \(x^2\) is even
\item
  \(\forall x \in \mathbb{Z}, {} x-1 < \left \lfloor{x}\right \rfloor\)
\item
  \(\forall a,b,c \in \{\texttt{True}, \texttt{False}\}, {} \neg a \lor \neg(b \land \neg c) \equiv a \implies (b \implies c)\)
\item
  \(\forall a,b,c \in \mathbb{Z}\), if \(a \mid b\) and \(b \mid c\),
  then \(a \mid c\)
\end{itemize}

\subsection{Forall, Exhaustive}\label{forall-exhaustive}

However, we can also prove ``for all'' statements by exhaustively
looking at every element in \(S\) and checking if \(P(x)\) holds.

Fittingly, this is called a \textbf{Proof by Exhaustion}.

A good example is the set problem you've already seen:

\begin{itemize}
\tightlist
\item
  \(F = \{\)Erik, José, Nicoleta, Aksana\(\}\)
\item
  \(V = \{\)Aksana, Erik\(\}\) is the set of your friends who are
  vegetarian
\item
  \(N = \{\)Aksana\(\}\) is the set of your friends who are vegan
\item
  We want to prove: \(\forall x \in F,{} x \in N \implies x \in V\)
\end{itemize}

\subsection{Proof by Cases}\label{proof-by-cases}

A \textbf{Proof by Cases} is a kind of proof by exhaustion. You are
breaking the set you are proving something about into smaller sets.

A good example is when we proved the definition of absolute value.

\(|x| = sgn(x) \cdot x\)

\(sgn(x) = \begin{cases}
    -1 & x<0\\
    0 & x =0\\
    1 & x>0
    \end{cases}\)

\subsection{Proof by Cases}\label{proof-by-cases-1}

A \textbf{Proof by Cases} is a kind of proof by exhaustion. You are
breaking the set you are proving something about into smaller sets.

A good example is when we proved the definition of absolute value.

\(|x| = sgn(x) \cdot x\)

\(sgn(x) = \begin{cases}
    -1 & x<0\\
    0 & x =0\\
    1 & x>0
    \end{cases}\)

Case 1: \(x > 0\)

\(\begin{array}{l l l}
      1. & x > 0   & \textrm{Case Assertion} \\
      2. & sgn(x) = 1  & \textrm{Signum Def.} \\
      3. & sgn(x) \cdot x = x & \textrm{Multiply both sides by } x \\
      4. & |x| = x & \textrm{Applied Def. of Absolute Value to Line 1} \\
      5. & sgn(x) = |x| & \textrm{Compared Lines 3 and 5} \\
    \end{array}\)

\subsection{Proof by Cases Cont.}\label{proof-by-cases-cont.}

Case 2: \(x = 0\)

\(\begin{array}{l l l}
      1. & x = 0   & \textrm{Case Assertion} \\
      2. & sgn(x) = 0  & \textrm{Signum Def.} \\
      3. & sgn(x) \cdot x = 0 & \textrm{Multiply both sides by } x \\
      4. & |x| = 0 & \textrm{Applied Def. of Absolute Value to Line 1} \\
      5. & sgn(x) = |x| & \textrm{Compared Lines 3 and 5} \\
    \end{array}\)

Case 3: \(x < 0\)

\(\begin{array}{l l l}
      1. & x < 0   & \textrm{Case Assertion} \\
      2. & sgn(x) = -1  & \textrm{Signum Def.} \\
      3. & sgn(x) \cdot x = -x & \textrm{Multiply both sides by } x \\
      4. & |x| = -x & \textrm{Applied Def. of Absolute Value to Line 1} \\
      5. & sgn(x) = |x| & \textrm{Compared Lines 3 and 5} \\
    \end{array}\)

\subsection{Proof of Existence}\label{proof-of-existence}

Sometimes you just need to prove the existence of something.

\(\exists x \in S, {} P(x)\).

Again, like we discussed in class, you can show the existence of
something by looking at every element of the set and finding an \(x\)
such that \(P(x)\) is true.

Let's go back to our food example:

\begin{itemize}
\tightlist
\item
  \(F = \{\)Erik, José, Nicoleta, Aksana\(\}\)
\item
  \(V = \{\)Aksana, Erik\(\}\) is the set of your friends who are
  vegetarian
\item
  \(N = \{\)Aksana\(\}\) is the set of your friends who are vegan
\item
  We want to prove: \(\exists x \in F,{} x \in N \implies x \in V\)
\end{itemize}

\subsection{Proof of Existence - Constructive
Proof}\label{proof-of-existence---constructive-proof}

Sometimes you will have to prove the existence of more complicated
things. You might have to \emph{construct} a solution and then prove
it's a valid solution.

This can also be called a \textbf{constructive proof} or \textbf{proof
by construction} because you are literally constructing your own
example.

\subsection{Proof by Construction -
Example}\label{proof-by-construction---example}

For example, you might want to argue the usefulness of exponential
encryption by proving that exponential encryption can be decrypted:

\(\exists e,d,x,N \in \mathbb{Z}, {} x^{e \cdot d} \bmod N = x\)

You can do this by using RSA as an example:

\subsubsection{Construction}\label{construction}

\begin{itemize}
\tightlist
\item
  \(N = p \cdot q\)
\item
  \(e \cdot d = 1 \bmod (p-1)(q-1)\)
\end{itemize}

\subsubsection{Proof}\label{proof}

Plug in \(N = p \cdot q\) and \(e \cdot d = 1 \bmod (p-1)(q-1)\) to
demonstrate \(x^{e \cdot d} \bmod N = x\).

\subsection{Proof by Induction}\label{proof-by-induction}

\subsection{(Strong) Induction}\label{strong-induction}

\begin{itemize}
\tightlist
\item
  Induction is a type of proof we can do on recursively defined
  functions and sets
\item
  Say we are trying to prove \(R(x)\) holds in a recursively defined set
  \(S = \{S_0, S_1, S_2, \ldots \}\)
\item
  We can prove this by:

  \begin{enumerate}
  \def\labelenumi{\arabic{enumi}.}
  \tightlist
  \item
    Showing \(R(x)\) holds for the base case(s) of \(S\)
  \item
    Assuming \(R(k)\) holds for all \(k < n\) in the recursive rule,
    showing that it also holds for step \(n\). In other words, we're
    showing
    \(\big(R(S_0) \land R(S_1) \land \ldots \land R(S_{n-2}) \land R(S_{n-1})\big) \implies R(S_{n})\).\\
  \end{enumerate}
\item
  Remember that many things can be defined recursively, so even though
  \(x \in S\), \(x\) isn't necessarily a single element. \(x\) can also
  be a set/function/mapping etc! Think about our recursive powersets
  definition.
\end{itemize}

\subsection{Let's Start With an
Example}\label{lets-start-with-an-example}

\subsection{Example with Template}\label{example-with-template}

Recall the Fibonacci series\ldots{}

\begin{itemize}
\tightlist
\item
  The base case: \(F(0) = 0, F(1) = 1\)
\item
  Recursive rule: For \(n > 1\), \(F(n) = F(n-1) + F(n-2)\).
\end{itemize}

We want to prove the following about the sum of the first \(n\) numbers
of the series:

\(\forall n \in \mathbb{N}, F(0) + F(1) + F(2) + \ldots + F(n-1) + F(n) = F(n+2) - 1\).

\subsection{Example with Template}\label{example-with-template-1}

S1. State the `for all' statement that you want to prove:

\begin{itemize}
\tightlist
\item
  \(\forall n \in \mathbb{N}, \sum_{i=0}^n F(i) = F(n+2) - 1\).
\end{itemize}

S2. Say ``we prove this by induction on'' and state the induction
parameter.

\begin{itemize}
\tightlist
\item
  We prove this by induction on \(n\).
\end{itemize}

\subsection{Example with Template}\label{example-with-template-2}

S3. Prove the base case(s).

\begin{itemize}
\tightlist
\item
  \(n=0\):

  \begin{itemize}
  \tightlist
  \item
    \(F(0) = F(2) - 1\)
  \item
    \(0 = 0\) \(\square\)
  \end{itemize}
\item
  \(n = 1\):

  \begin{itemize}
  \tightlist
  \item
    \(F(0) + F(1) = F(3) - 1\)
  \item
    \(1 = 2 - 1\) \(\square\)
  \end{itemize}
\end{itemize}

\subsection{Example with Template}\label{example-with-template-3}

S4. Write Induction Step.

\begin{itemize}
\tightlist
\item
  For a given \(n > 1\),
\end{itemize}

S5. State the Induction Hypothesis (IH)

\begin{itemize}
\tightlist
\item
  I can assume for all \(1 \leq k \leq n\) that
  \(\sum_{i=0}^k F(i) = F(k+2) - 1\),
\end{itemize}

S6. State what you are going to prove about your specific value of \(x\)
that was given to you in S4:

\begin{itemize}
\tightlist
\item
  and I want to prove \(\sum_{i=0}^n F(i) = F(n+2) - 1\).
\end{itemize}

\subsection{Example with Template}\label{example-with-template-4}

S7. Do the proof for the specific \(x\).

\subsection{Example with Template}\label{example-with-template-5}

S7. Do the proof for the specific \(x\).

\(\begin{array}{l l l}
        1. & \forall 1 \leq k \leq n, {} \sum_{i=0}^k F(i) = F(k+2) - 1 & \textrm{IH} \\
        2. & \textrm{Let } k = n - 1, \textrm{ then } \sum_{i=0}^{n-1} F(i) = F(n-1+2) - 1 & \textrm{Applied IH} \\
        3. & \sum_{i=0}^{n-1} F(i) = F(n+1) - 1 & \textrm{Rewrote Line 2}\\
        4. & \sum_{i=0}^{n-1} F(i) + F(n) = F(n+1) - 1 + F(n) & \textrm{Added } F(n) \textrm{ to both sides.}\\
        5. & F(n+2) = F(n+1) + F(n) & \textrm{Fibonacci Def.}\\
        6. & \sum_{i=0}^{n-1} F(i) + F(n) = F(n+2) - 1 & \textrm{Plugged 5. into 4.}\\
        7. & \sum_{i=0}^n F(i) = \sum_{i=0}^{n-1} F(i) + F(n) & \textrm{Def. of Sum} \\
        8. & \sum_{i=0}^{n} F(i) = F(n+2) - 1 & \textrm{Plugged 7. into 6.} \square
    \end{array}\)

\subsection{Example with Template}\label{example-with-template-6}

S8. Declare victory.

\begin{itemize}
\tightlist
\item
  Therefore, we have proved
  \(\forall n \in \mathbb{N}, F(0) + F(1) + F(2) + \ldots + F(n-1) + F(n) = F(n+2) - 1\).
\end{itemize}

\subsection{Tips for Proving Something by Induction
(S7)}\label{tips-for-proving-something-by-induction-s7}

\begin{itemize}
\tightlist
\item
  Your Inductive Hypothesis (Step 5) should be line 1 in your proof.
\item
  The recursive definition of any structures you're using (e.g.~sum,
  factorial, exponents, etc.) should be the next lines of your proof.
\item
  Step 6 is what the \emph{last} line of your proof should be.
\item
  To start, think about how you can use your inductive hypothesis and
  recursive definitions to get to step 6. (Usually this means plugging a
  smaller version of \(x\) into your inductive hypothesis and using your
  recursive definitions to rewrite it.)
\end{itemize}
