% Options for packages loaded elsewhere
\PassOptionsToPackage{unicode}{hyperref}
\PassOptionsToPackage{hyphens}{url}
\documentclass[
]{article}
\usepackage{xcolor}
\usepackage{amsmath,amssymb}
\setcounter{secnumdepth}{-\maxdimen} % remove section numbering
\usepackage{iftex}
\ifPDFTeX
  \usepackage[T1]{fontenc}
  \usepackage[utf8]{inputenc}
  \usepackage{textcomp} % provide euro and other symbols
\else % if luatex or xetex
  \usepackage{unicode-math} % this also loads fontspec
  \defaultfontfeatures{Scale=MatchLowercase}
  \defaultfontfeatures[\rmfamily]{Ligatures=TeX,Scale=1}
\fi
\usepackage{lmodern}
\ifPDFTeX\else
  % xetex/luatex font selection
\fi
% Use upquote if available, for straight quotes in verbatim environments
\IfFileExists{upquote.sty}{\usepackage{upquote}}{}
\IfFileExists{microtype.sty}{% use microtype if available
  \usepackage[]{microtype}
  \UseMicrotypeSet[protrusion]{basicmath} % disable protrusion for tt fonts
}{}
\makeatletter
\@ifundefined{KOMAClassName}{% if non-KOMA class
  \IfFileExists{parskip.sty}{%
    \usepackage{parskip}
  }{% else
    \setlength{\parindent}{0pt}
    \setlength{\parskip}{6pt plus 2pt minus 1pt}}
}{% if KOMA class
  \KOMAoptions{parskip=half}}
\makeatother
\setlength{\emergencystretch}{3em} % prevent overfull lines
\providecommand{\tightlist}{%
  \setlength{\itemsep}{0pt}\setlength{\parskip}{0pt}}
\usepackage{bookmark}
\IfFileExists{xurl.sty}{\usepackage{xurl}}{} % add URL line breaks if available
\urlstyle{same}
\hypersetup{
  pdftitle={Nondeterministic Finite Automata},
  pdfauthor={Alyssa Lytle},
  hidelinks,
  pdfcreator={LaTeX via pandoc}}

\title{Nondeterministic Finite Automata}
\author{Alyssa Lytle}
\date{January 22, 2025}

\begin{document}
\maketitle

\section{Nondeterministic Finite
Automata}\label{nondeterministic-finite-automata}

\subsection{Determinism}\label{determinism}

``When a machine is in a given state and reads the next input symbol, we
know what the next state will be--it is \emph{determined}.'' \footnote{\begin{itemize}
  \tightlist
  \item
    Sipser, Michael. ``Introduction to the Theory of Computation.'' ACM
    Sigact News 27.1 (1996): 27-29
  \end{itemize}}

\subsection{Nondeterminism}\label{nondeterminism}

\begin{itemize}
\item
  ``A state is not uniquely determined by its current state.'' - Kozen
  \footnote{\begin{itemize}
    \tightlist
    \item
      Kozen, Dexter C. Automata and computability. Springer Science \&
      Business Media, 2007.
    \end{itemize}}
\item
  ``\ldots the power to be in several states at once.'' - Hopcroft et
  al.\footnote{\begin{itemize}
    \tightlist
    \item
      Hopcroft, John E., Rajeev Motwani, and Jeffrey D. Ullman.
      ``Introduction to automata theory, languages, and computation.''
      Acm Sigact News 32.1 (2001): 60-65.
    \end{itemize}}
\item
  ``\ldots several choices may exist for the next state at any point.''
  - Sipser \footnote{\begin{itemize}
    \tightlist
    \item
      Sipser, Michael. ``Introduction to the Theory of Computation.''
      ACM Sigact News 27.1 (1996): 27-29
    \end{itemize}}
\end{itemize}

\subsection{Why???}\label{why}

\begin{itemize}
\item
  Represents real-life situations where there's not
  enough/incomplete/unpredictable/unreliable information about external
  forces and how they impact the state.
\item
  Can be a useful tool in computation. Some algorithms rely on
  nondeterminism for more efficient solutions.
\item
  Nondeterministic definitions can be simpler/more concise.
\end{itemize}

\subsection{Nondeterministic Finite
Automata}\label{nondeterministic-finite-automata-1}

A Nondeterministic Finite Automaton (NFA) has a similar 5-tuple
definition as the Deterministic Finite Automata (DFA) we've seen so far:
\((Q, \Sigma, \delta, s, F)\), but some components are defined
differently.

Thinking of our 5-tuple definition and the definition of nondeterminism,
what components do you think are different?

\subsection{Nondeterministic Finite
Automata}\label{nondeterministic-finite-automata-2}

We still have:

\begin{itemize}
\tightlist
\item
  \(Q\): A finite set of states
\item
  \(\Sigma\): A finite alphabet
\item
  \(F \subseteq Q\): A set of accept states
\end{itemize}

But we can have

\begin{itemize}
\item
  a \emph{set} of start states!
\item
  \(\delta\) can transition to a \emph{set} of possible next-states! You
  also don't \emph{have} to have a transition defined for every state,
  input combination!
\end{itemize}

\subsection{Nondeterministic Finite Automata - Formal
Definition}\label{nondeterministic-finite-automata---formal-definition}

A \emph{nondeterministic finite automaton} (NFA) is a five-tuple:

\[N = (Q, \Sigma, \Delta, S, F)\]

where

\begin{itemize}
\tightlist
\item
  \(Q\): A finite set of states
\item
  \(\Sigma\): A finite alphabet
\item
  \(\Delta\): a function \(Q \times \Sigma \to 2^Q\)
\item
  \(S \subseteq Q\): A \emph{set} of start states
\item
  \(F \subseteq Q\): A set of accept states
\end{itemize}

Where \(2^Q\) is the \emph{power set} of \(Q\).
(\(\{A \mid A \subseteq Q\}\))

\subsection{Example}\label{example}

Draw an NFA over the alphabet \(\{a,b\}\) such that it accepts:

\[A = \{w \in \{a,b\}^* \mid \textrm{the last symbol is } a \}\]

E.g. it accepts \(ababba\) and \(aaa\) but not \(aab\) or \(babab\).

\subsection{Another feature: Epsilon
Transitions}\label{another-feature-epsilon-transitions}

\begin{itemize}
\item
  \(\epsilon\)-transitions can be useful in simplifying representation
  of a diagram.
\item
  Essentially, they give us transitions over \emph{no} input (aka the
  empty string \(\epsilon\))
\end{itemize}

\subsection{Another example}\label{another-example}

Draw an NFA over the alphabet \(\{a,b\}\) such that it accepts:

\[A = \{w \in \{a,b\}^* \mid w \textrm{has } 2m \textrm{ or } 3m \textrm{ } a's \}\]

\subsection{What does ``acceptance''
mean?}\label{what-does-acceptance-mean}

A nondeterministic automaton is said to \emph{accept} its input \(w\) if
there exists \emph{at least} one computation path on input \(w\) from a
start state to an accept state.

\subsection{Some properties}\label{some-properties}

\begin{itemize}
\item
  Every DFA can be expressed as an NFA. (Reasonable.)
\item
  Every NFA can be expressed as a DFA. (A little more complicated to
  think about\ldots)
\end{itemize}

\subsection{Every DFA can be expressed as an
NFA}\label{every-dfa-can-be-expressed-as-an-nfa}

Let's take an example DFA from a previous class\ldots{}

\[A = \{w \in \{a,b\}^* \mid w \textrm{ has odd length} \}\]

\subsection{Every NFA can be expressed as an
DFA}\label{every-nfa-can-be-expressed-as-an-dfa}

Let's use our earlier example of our NFA that accepts

\[A = \{w \in \{a,b\}^* \mid \textrm{the last symbol is } a \}\]

\subsection{Basic Procedure}\label{basic-procedure}

To convert NFA \(N = (Q_N, \Sigma, \Delta_N, S_N, F_N)\) to DFA
\(M = (Q_M, \Sigma, \delta_M, s_M, F_M)\),

From a high-level: Set the states of \(M\) to be the \emph{powerset} of
the states of \(N\), and follow the rest of the construction logically
from there.

Formally:

\begin{itemize}
\tightlist
\item
  \(Q_M = 2^{Q_N}\)
\item
  \(\delta_M(A,a) = \bigcup_{q \in A}\Delta_N(q,a)\)
\item
  \(s_M = S_N\)
\item
  \(F_M = \{A \subseteq Q_N | A \cap F_N \neq \emptyset\}\)
\end{itemize}

\section{Sources}\label{sources}

\end{document}
