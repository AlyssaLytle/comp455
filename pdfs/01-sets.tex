\section{Review: Set Notation and
Operations}\label{review-set-notation-and-operations}

\subsection{Sets - Definitions}\label{sets---definitions}

A \textbf{\emph{set}} is an unordered collection of objects.

The following are sets:

\begin{itemize}
\tightlist
\item
  \(\{1,2,3\}\)
\item
  ``all multiples of 7'\,'
\item
  \{apples, \(7\), \(\texttt{True}\)\}
\end{itemize}

Sets don't inherently have an order.

\subsection{Sets - Terminology}\label{sets---terminology}

A set is a \textbf{\emph{finite set}} if it has a finite number of
elements.

Any set that is not finite is an \textbf{\emph{infinite set}}.

Let \(A\) be a finite set. The number of different elements in \(A\) is
called its \textbf{\emph{cardinality}}.

The cardinality of a finite set is denoted \(|A|\).

\subsubsection{Examples}\label{examples}

\begin{quote}
\(\{1,2,3\}\) is a finite set. Its cardinality is \(3\).
\end{quote}

\begin{quote}
``all multiples of 7'' is an infinite set.
\end{quote}

\subsection{Sets - Notation}\label{sets---notation}

\(a \in A\) means \(a\) is an element of \(A\).

\(a \notin A\) means \(a\) is \emph{not} an element of \(A\).

\subsubsection{Example}\label{example}

\begin{quote}
Let \(A = \{apples, bananas, oranges\}\)
\end{quote}

\begin{quote}
``apples'' \(\in A\)
\end{quote}

\begin{quote}
``blueberries'' \(\notin A\)
\end{quote}

\subsection{Sets - Notation}\label{sets---notation-1}

Sets are commonly expressed using \emph{set notation}.

Within braces, we can write a rule consisting of a function, a vertical
bar, and a set to which the function is applied.

\subsection{Sets - Notation}\label{sets---notation-2}

\subsubsection{Example}\label{example-1}

\begin{quote}
We can express the set \(\{\frac{1}{2}, \frac{1}{4}, \frac{1}{8} \}\)
as\ldots{}
\end{quote}

\begin{quote}
\end{quote}

\subsection{Subsets}\label{subsets}

\(A\) is a \textbf{\emph{subset}} of \(B\) if and only if every element
of \(A\) is an element of \(B\).

Can also be written: \(A \subseteq B\)

\subsubsection{Examples}\label{examples-1}

\begin{quote}
Let \(A = \{1,3,5\}\) and \(B = \{1,2,3,4,5\}\).
\end{quote}

\begin{quote}
\begin{quote}
\(A \subseteq B\).
\end{quote}
\end{quote}

\begin{quote}
Let \(A = \{1,2,3\}\) and \(B = \{3,2,1\}\).
\end{quote}

\begin{quote}
\begin{quote}
\(A \subseteq B\) and \(B \subseteq A\).
\end{quote}
\end{quote}

\subsection{Equality}\label{equality}

\(A = B\) if and only if every element of \(A\) is an element of \(B\)
and conversely every element of \(B\) is an element of \(A\). That is,
\(A \subseteq B\) and \(B \subseteq A\).

\subsubsection{Example}\label{example-2}

\begin{quote}
Let \(A = \{1,2,3\}\) and \(B = \{3,2,1\}\). \(A \subseteq B\) and
\(B \subseteq A\), so \(A=B\)
\end{quote}

\subsection{Common Sets}\label{common-sets}

\begin{itemize}
\tightlist
\item
  There are some standard symbols that represent specific sets you will
  see:
\item
  The set of \textbf{Natural Numbers} \(\mathbb{N}\) is the set of all
  whole numbers \(\geq 0\), \(\{0,1,2,3,4,\ldots\}\).*
\item
  The set of \textbf{Integers} \(\mathbb{Z}\) is the set of all whole
  numbers, \(\{\ldots, -3, -2, -1, 0, 1, 2, 3, \ldots\}\).
\item
  The set of \textbf{Rational Numbers} \(\mathbb{Q}\) are numbers that
  can be represented as a quotient of whole numbers,
  \(\{\frac{p}{q} \mid p,q \in \mathbb{Z}\}\)
\item
  The set of \textbf{Real Numbers} \(\mathbb{R}\) is all \emph{real}
  numbers.
\end{itemize}

\subsection{Tuples}\label{tuples}

A \(k\)-\textbf{\emph{tuple}} is an ordered sequence of \(k\) elements,
which we write down in parentheses, \((a_1,a_2,\ldots,a_k)\).

\begin{quote}
The most common tuple seen in math is the coordinate pair \((x,y)\) on a
graph.
\end{quote}

\begin{quote}
A 2-tuple is commonly called an \emph{ordered pair}.
\end{quote}

Two tuples are equal if and only if all of their corresponding elements
are equal. \((a_1,a_2,\ldots,a_k)\) iff for all \(i\in[1,\ldots,k]\) we
have \(a_i=b_i\).

\subsection{Concatenation}\label{concatenation}

\textbf{\emph{Concatenation}} is used to join two strings or lists by
putting all elements of the second string behind all elements of the
first.

\subsubsection{Example}\label{example-3}

\begin{quote}
The concatenation of ``hello'' and ``world'' is ``helloworld''.
\end{quote}

\subsection{Range}\label{range}

\([a,b]\) is the set of whole numbers \(\geq a\) and \(\leq b\).

\((a,b)\) is the set of whole numbers \(> a\) and \(< b\).

\subsubsection{Examples}\label{examples-2}

\begin{quote}
\([1,5] = \{1,2,3,4,5\}\)
\end{quote}

\begin{quote}
\((1,5) = \{2,3,4\}\)
\end{quote}

\begin{quote}
\([1,5) = \{1,2,3,4\}\)
\end{quote}

\section{Set Operations}\label{set-operations}

\subsection{Set Operations}\label{set-operations-1}

\begin{itemize}
\tightlist
\item
  \(a \in B\) means \(a\) is an element of \(B\).
\end{itemize}

\subsection{Set Operations}\label{set-operations-2}

\begin{itemize}
\item
  \(a \notin B\) means \(a\) is \emph{not} an element of \(B\).
\item
  Note that technically, \(a \in B\) and \(a \notin B\) are predicates!
  They take an element and a set as input and give True or False as an
  output.
\end{itemize}

\subsection{Subset}\label{subset}

Let \(A\) and \(B\) be sets. We say that \(A\) is a \textbf{subset} of
\(B\) if and only if every element of \(A\) is an element of \(B\).

We write \(A \subseteq B\) to denote the fact that \(A\) is a subset of
\(B\).

\subsection{Equality}\label{equality-1}

\subsubsection{Using Predicate Logic}\label{using-predicate-logic}

\begin{itemize}
\tightlist
\item
  Remember this?
\item
  For all sets \(A\) and \(B\), \(A = B\) if and only if every element
  of \(A\) is an element of \(B\) and every element of \(B\) is an
  element of \(A\)
\item
  \(\forall A,B, A = B \leftrightarrow (A \subseteq B\) and
  \(B \subseteq A)\)
\end{itemize}

\subsection{Complement}\label{complement}

The complement of a set \(A\), denoted \(A^\mathcal{C}\) is the set of
all elements in the universe \(U\) that are \emph{not} in \(A\).

\subsubsection{Using Set Notation}\label{using-set-notation}

\begin{quote}
\(A^\mathcal{C} = \{x  |  x \notin A\}\)
\end{quote}

\subsubsection{Using Predicate Logic}\label{using-predicate-logic-1}

\begin{quote}
\(\forall a,  a \in A^\mathcal{C} \leftrightarrow a \notin A\)
\end{quote}

\begin{quote}
or, equivalently,
\(\forall a \in U,  a \notin A^\mathcal{C} \leftrightarrow a \in A\)
\end{quote}

\subsection{Intersection}\label{intersection}

\(A \cap B\) are the elements that are both in \(A\) and \(B\).

\subsubsection{Using Set Notation}\label{using-set-notation-1}

\begin{quote}
\(A \cap B = \{ x  |  x \in A \land x \in B \}\)
\end{quote}

\subsubsection{Using Predicate Logic}\label{using-predicate-logic-2}

\begin{quote}
\(\forall A, B, x,  x \in A \cap B \leftrightarrow (x \in A \land x \in B)\)
\end{quote}

\subsection{Union}\label{union}

\(A \cup B\) are the elements that are either in \(A\) or \(B\).

\subsubsection{Using Set Notation}\label{using-set-notation-2}

\begin{quote}
\(A \cup B = \{ x  |  x \in A \lor x \in B \}\)
\end{quote}

\subsubsection{Using Predicate Logic}\label{using-predicate-logic-3}

\begin{quote}
\(\forall A, B, x,  x \in A \cup B \leftrightarrow (x \in A \lor x \in B)\)
\end{quote}

\subsection{Difference}\label{difference}

The \textbf{difference} of sets \(A\) and \(B\) is the set that contains
all elements in \(A\) that are not in \(B\).

\subsubsection{Using Set Notation}\label{using-set-notation-3}

\begin{quote}
\(A - B = A \backslash B =\{\, x  |  x \in A \land x \notin B \}\)
\end{quote}

\subsubsection{Using Predicate Logic}\label{using-predicate-logic-4}

\begin{quote}
\(\forall A, B, x,  x \in A - B \leftrightarrow x \in A \land x \notin B\)
\end{quote}

\subsection{Difference Cont.}\label{difference-cont.}

\subsubsection{Example}\label{example-4}

\begin{quote}
Let \(A=\{1,3,5,7\}\) and \(B=\{4,5,6,7,8\}\).
\end{quote}

\begin{quote}
\(A-B = \{1,3\}\).
\end{quote}

\subsubsection{Example}\label{example-5}

\begin{quote}
Let \(C=\{\bigcirc,\diamondsuit,\Box,\heartsuit\}\) and
\(e = \heartsuit\).
\end{quote}

\begin{quote}
\(C-\{e\} = \{\bigcirc, \diamondsuit, \Box\}\).
\end{quote}

\subsection{Xor}\label{xor}

\subsubsection{Using Set Notation}\label{using-set-notation-4}

\begin{quote}
\(A \oplus B = \{x  |  x \in A \oplus x \in B\}\)
\end{quote}

\subsubsection{Using Predicate Logic}\label{using-predicate-logic-5}

\begin{quote}
\(\forall A,B,x,  x \in A \oplus B \leftrightarrow x \in A \oplus x \in B\)
\end{quote}

\subsection{Cartesian Product}\label{cartesian-product}

The \textbf{cartesian product} of \(A\) and \(B\),

\subsubsection{Using Set Notation}\label{using-set-notation-5}

\begin{quote}
\(A \times B = \{(a,b)| \forall a \in A\), \(\forall b \in B\}\)
\end{quote}

\subsubsection{Using Predicate Logic}\label{using-predicate-logic-6}

\begin{quote}
\(\forall A,B, a,b,  ((a,b) \in A \times B) \leftrightarrow (a \in A \land b \in B)\)
\end{quote}

\subsection{Powerset}\label{powerset}

The powerset of a set \(A\), denoted \(\mathcal{P}(A)\) is the set of
all subsets of \(A\)

\subsubsection{Using Set Notation}\label{using-set-notation-6}

\begin{quote}
\(\mathcal{P}(A) = \{ S  |  S \subseteq A\}\)
\end{quote}

\begin{itemize}
\tightlist
\item
  \(|\mathcal{P}(A)| = 2^{|A|}\)
\end{itemize}

\subsection{Next}\label{next}

There's a lesson on Gradescope!
